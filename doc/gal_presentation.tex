\documentclass[table, czech]{beamer}
\usepackage[utf8]{inputenc}
\usepackage[czech]{babel}
\usepackage{array}
\usepackage{amsthm}
\usepackage{graphicx}
\usepackage{listings}
\usepackage{subfig}
\usepackage{float}
\usepackage{multirow}
\usepackage{booktabs}
\usepackage{color}
\usepackage{alltt}

% additional commands
\newcommand{\setHelper}[1]{\left\lbrace #1 \right\rbrace}

% beamer settings
\usetheme{CambridgeUK}
\setbeamertemplate{footline}[frame number]
\setbeamercolor{block title}{fg=cambridgedarkblue,bg=white}
\setbeamercolor{block body}{fg=black,bg=white}
\newtranslation[to=Czech]{Definition}{Definice}


% title page
\title{Algoritmus pro hledání maximálních nezávislých množin}
\author{Milan Munzar, Jakub Sochor}
\institute[FIT]{Fakulta informačních technologií VUT v Brně}

\begin{document}

\frame[plain] { \titlepage }


\frame
{
    \frametitle{Maximální nezávislé množiny}
}

\frame
{
    \frametitle{Algoritmus pro hledání maximálních nezávislých množin}
}

\frame
{
    \frametitle{Paralelizace algoritmu}
}

\frame
{
    \frametitle{Použité prostředky}
}


\frame
{
    \frametitle{Dosažené výsledky}
}

\end{document} 

